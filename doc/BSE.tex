\documentclass{article}
\usepackage[hmargin=2cm,vmargin=3cm]{geometry}
\usepackage{amsmath}
\usepackage{graphicx}
\usepackage{indentfirst}
\title{A Time-Dependent Formulation for the Bethe-Salpeter Equation}
\author{Songchen}
\date{\today}
\begin{document}
\maketitle
\paragraph{Abstract} Some Abstract.
\section{Background}
Understanding the optical absorption spectra of large nanosystems \emph{via} appropriate theoretical approaches is indispensable in designing advanced materials and electronic devices. Time-dependent density functional theory (TDDFT) has been extremely successful for describing excited state properties for small molecules, but often fail for complex extended systems, where multiple excitations, as well as low-lying excitonics may appear.

An alternative approach to investigate the excited state structures is many-body perturbation theory (MBPT). In general, the most common tools from MBPT is \emph{GW} approximation for quasiparticle excitations and Bethe-Salpeter equation (BSE) for particle-hole excitations. The BSE is often formulated with a eigenvalue problem, analogous to the Casida equation in TDDFT. However, solving the eigenvalue problem usually scales as at least $O(N^6)$, which prohibits the application in large systems.

Since the time-dependent formulation is in principle much cheaper and extensible, considerable efforts have been made on developing novel time-dependent approach that inspired by MBPT relations. These attempts include deriving exchange-correlation kernel $f_{\rm xc}$ from BSE. However, since the direct interaction block in BSE is intrinsically non-local and frequency-dependent, approximating it in terms of TDDFT has not been satisfactory.

Recently, inspired by the well-known connection between CIS and TDHF, we have proposed a time-dependent Bethe-Salpeter approach, where we used a modified exchange operator to approximately represent the direct interaction block with a scalar screening constant $\varepsilon$. However, determining the screening constant may require some \emph{a priori} knowledge for the system.

In this paper, we present a fully first-principle time-dependent formulation for Bethe-Salpeter equation, where the modified exchange opeartor can account for both static and dynamic screening. The accuracy has been demonstrated with a 1D Hubbard model.
\section{Theory}
\subsection{Bethe-Salpeter Equation}
It is well known that the original BSE, which can be easily derived from functional derivative approach, take the form

$$
L(1,2,1',2')=L_0(1,2,1',2')+L_0(1,3,1',3') \Xi(3',4',3,4)L(4,2,4',2')
$$

Where $L$ is the two-particle correlation function defined by $L(1,2,1',2') = G(1,1')G(2,2')-G_2(1,2,1,2')$, $L_0$ is the non-interacting correlation function $L_0(1,2,1',2')=G(1,2')G(2,1')$, and $\Xi(3',4',3,4)$ is the exchange correlation kernel, formally defined by the functional derivative of self-energy with respect to Green function

$$
\Xi(3',4',3,4) = \frac{\delta\Sigma(3',3)}{\delta G(4,4')}
$$

As long as $GW$ approximation is valid, and assuming that $W$ does not depend on $G$, the exchange-correlation kernel $\Xi$ can be approximated by

$$
\Xi(3',4',3,4)=-i\delta(3',3)\delta(4',4)v(3',4')+i\delta(3',4)\delta(4',3)W(3',4')
$$
\subsection{Eigenvalue Formulation}
\subsubsection{Static $W$}
The above equation can be transform to frequency domain:

$$
L(\omega_1,\omega_2)=L_0(\omega_1,\omega_2)+\frac{L_0(\omega_1,\omega_2)}{2\pi}\int\mathrm d\omega_3[v-W(\omega_2-\omega_3)]L(\omega_1,\omega_3)
$$

Due to the coupling in frequency, it cannot be further simplified. However, if one take $W(\omega)\approx W(0)$, we can integrate out the second frequency argument of $L$, and obtain

$$
L(\omega)=L_0(\omega)-iL_0(\omega)vL(\omega)+iL_0(\omega)WL(\omega)
$$

Then, by inserting complete sets of quasi-particle states, we are able to formulate a eigenvalue problem:

$$
\begin{pmatrix}A&B\\-B&-A\end{pmatrix}
\begin{pmatrix}X\\Y\end{pmatrix}=\omega
\begin{pmatrix}X\\Y\end{pmatrix}
$$

Where the matrix block reads $A=D+2K^X-K^D$, and $B=2K^X-K^D$. These matrix elements are given by

$$
\begin{aligned}
D_{ia,jb}&=(\varepsilon_a-\varepsilon_i)\delta_{ia,jb}\\
K^X_{ia,jb}&=\langle ia|v|jb\rangle\\
K^D_{ia,jb}&=\langle ab|W(0)|ij\rangle
\end{aligned}
$$

\subsubsection{Dynamic $W$}

No simple eigenvalue problem can be proposed for dynamic $W$. However, assuming that the electron-hole excitation energies is well separated, we can approximate the dynamical effect through a convolution of $W$, which results in replacing the direct interaction matrix by:

$$
\begin{aligned}
K^D_{ia,jb}(\omega)&=\int\mathrm dr\mathrm dr'a(r)b(r)i(r')j(r')\frac{i}{2\pi}\\
&\times\int\mathrm d\omega'e^{-i\omega' 0^+}W(r,r',\omega')\left[\frac1{\omega-(\varepsilon_b-\varepsilon_i)-\omega'+i0^+}+\frac1{\omega-(\varepsilon_a-\varepsilon_j)+\omega'+i0^+}\right]
\end{aligned}
$$

In practice, this can often be done by using a plasmon expansion of $W$ and performing the convolution analytically. The plasmon expansion usually takes the form

$$
W(r,r', \omega)= \sum_{l} \frac{W_{l}(r, r')}2 \left(\frac{\omega_{l}}{\omega-\omega_{l}}-\frac{\omega_{l}}{\omega+\omega_{l}}\right)
$$

The direct interaction block will thus be

$$
K^D_{ia,jb}(\omega)=-\int\mathrm dr\mathrm dr'a(r)b(r)i(r')j(r')\left[\sum_l\frac{W_{l}(r, r')}2 \left(\frac{\omega_{l}}{\omega_{l}-A}+\frac{\omega_{l}}{\omega_{l}-B}\right)\right]
$$

\subsection{Time-Dependent Formulation}
(Not completed.)
\section{Methods}
We consider a one-dimensional Hubbard model with $n=8$ atoms. The system is assumed to be closed-shell and is filled with $N=8$ electrons. The Hamiltonian reads

$$
H=\sum_{\langle lm\rangle,\sigma}T_{lm}c_{l\sigma}^{\dagger}c_{m\sigma}+ \sum_{l}n_{l\uparrow}n_{l\downarrow}
$$

Furthermore, we use an non-periodic and alternating hopping matrix:

$$
T_{l,l+1}=
\begin{cases}
\alpha&l=1,3,5,7\\
\beta&l=2,4,6\\
\end{cases}
$$

We first carried out closed-shell Hartree-Fock calculation and obtain independent particle orbitals and eigenvalues. We then use them as a basis to solve the BSE within eigenvalue formulation for both static and dynamic $W$, as well as to solve the BSE within time-dependent formulation for both static and (first order) dynamic $W$. Finally, we perform a CI calculation to validate all above results.

\section{Results}
\subsection{Comparison of Excitation Energies}

We first choose the parameter to be $\alpha=1.5$ and $\beta=1$. The closed-shell Hartree-Fock calculation gives us a set of energy levels which is symmetric around 0.5.

\begin{figure}[!ht]
    \centering
    \includegraphics[scale=0.5]{hf.png}
    \caption{Hartree-Fock Energies}
\end{figure}

Specifically, the HOMO is at $-0.28$ and the LUMO is at $1.28$, thus the band gap is $1.57$. We note that this moderate band gap represents semiconductors. 

We then investigate the eigenvalue problem with different version of approximations. We denote $K^{D}_{\rm Hartree}$ for $K^{D}_{ia,jb}=0$, $K^{D}_{\rm Fock}$ for $K^{D}_{ia,jb}=\langle ab|v|ij\rangle$, and $K^D_{\rm static}$, $K^D_{\rm dynamic}$ for the "real" BSE matrix.

\begin{figure}
    \centering
    \includegraphics[scale=0.6]{compare.png}
    \caption{Lowest Optical Excitation Calculated from Various Kind of Eigenvalue and Time-Dependent Approach}
\end{figure}

For all types of calculation (Hartree, Hartree-Fock, static BSE, dynamic BSE), the time-dependent calculation coincides the eigenvalue calculation. We note that the BSE results are consistently most accurate when compared to CI.

\section{Conclusion}
\end{document}
