\documentclass{article}
\usepackage[hmargin=2cm,vmargin=3cm]{geometry}
\usepackage{amsmath}
\usepackage{indentfirst}
\title{Time-Dependent Formulation for Bethe-Salpeter Equation}
\author{Songchen}
\date{February 2020}
\begin{document}
\maketitle
\paragraph{Abstract} Some Abstract.
\twocolumn
\section{Introduction}
(Not completed, just a placeholder.)
\subsection{TDDFT and its limitations}
\subsection{MBPT}
\subsection{Our Recent Progress}
\subsection{Closing Remarks}
\section{Theory}
\subsection{Bethe-Salpeter Equation}
It is well known that the original BSE take the form
$$
L(1,2,1',2')=L_0(1,2,1',2')+L_0(1,3,1',3') \Xi(3',4',3,4)L(4,2,4',2')
$$
In the context of $GW$ approximation, we approximate $\Xi$ by
$$
\Xi(3',4',3,4)=-i\delta(3',3)\delta(4',4)v(3',4')+i\delta(3',4)\delta(4',3)W(3',4')
$$
\subsection{Eigenvalue Formulation}
\subsubsection{Static $W$}
After we transform the BSE to frequency domain, it becomes
$$
L(\omega_1,\omega_2)=L_0(\omega_1,\omega_2)+\frac{L_0(\omega_1,\omega_2)}{2\pi}\int\mathrm d\omega_3[v-W(\omega_2-\omega_3)]L(\omega_1,\omega_3)
$$

Due to the coupling in frequency, it cannot be further simplified. However, if we take $W(\omega)\approx W(\omega=0)$, we can integrate out the second frequency argument of $L$, and obtain
$$
L(\omega)=L_0(\omega)-iL_0(\omega)vL(\omega)+iL_0(\omega)WL(\omega)
$$
Then, by inserting complete sets of quasi-particle states, we are able to formulate a eigenvalue problem:
$$
\begin{pmatrix}A&B\\-B&-A\end{pmatrix}
\begin{pmatrix}X\\Y\end{pmatrix}=\omega
\begin{pmatrix}X\\Y\end{pmatrix}
$$
$A=D+2K^X+K^D$, and $B=2K^X+K^D$. Then matrix elements are given by
$$
\begin{aligned}
D_{ia,jb}&=(\varepsilon_a-\varepsilon_i)\delta_{ia,jb}\\
K^X_{ia,jb}&=\langle ia|v|jb\rangle\\
K^D_{ia,jb}&=\langle ab|W(\omega=0)|ij\rangle
\end{aligned}
$$
\subsection{Dynamic $W$}
No similar eigenvalue problem can be proposed for dynamic $W$. However, assuming that the electron-hole excitation energies is well separated, we can approximate the dynamical effect through a convolution of $W$, which results in replacing the direct interaction matrix by:

$$
\begin{aligned}
K^D_{ia,jb}(\omega)&=\int\mathrm dr\mathrm dr'a(r)b(r)i(r')j(r')\frac{i}{2\pi}\\
&\times\int\mathrm d\omega'e^{-i\omega' 0^+}W(r,r',\omega')\left[\frac1{\omega-(\varepsilon_b-\varepsilon_i)-\omega'+i0^+}\right.\\
&\left.+\frac1{\omega-(\varepsilon_a-\varepsilon_j)+\omega'+i0^+}\right]
\end{aligned}
$$

In practice, this can often be done by using a plasmon expansion of $W$ and performing the convolution analytically.
\subsection{Time-Dependent Formulation}
(Not completed.)
\section{Methods}
\subsection{Systems}
We consider a one-dimensional Hubbard model with $n=8$ atoms. The system is assumed to be closed-shell and is filled with $N=8$ electrons. The Hamiltonian reads

$$
H=\sum_{\langle lm\rangle,\sigma}T_{lm}c_{l\sigma}^{\dagger}c_{m\sigma}+ \sum_{l}n_{l\uparrow}n_{l\downarrow}
$$

Furthermore, we use an non-periodic and alternating hopping matrix:

$$
T_{l,l+1}=
\begin{cases}
\alpha&l=1,3,5,7\\
\beta&l=2,4,6\\
\end{cases}
$$

Henceforth we set $\alpha=1.5$ and $\beta=1$.

\subsection{Procedures}
We first carried out closed-shell Hartree-Fock calculation and obtain independent particle orbitals and eigenvalues. We then use them as a basis to:

\begin{itemize}
    \item Solve the BSE within eigenvalue formulation for both static and dynamic $W$;
    \item Solve the BSE within time-dependent formulation for both static and (first order) dynamic $W$.
\end{itemize}

\section{Results}
\subsection{Comparison of Excitation Energies}
HF calculation gives us a HOMO at $-0.2832$ and LUMO at $1.2832$, thus the band gap is $1.5664$. Static eigenvalue formulation gives us $1.7421$, and the dynamic eigenvalue formulation gives us $1.7218$.
\section{Conclusion}
\end{document}
